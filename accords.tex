\documentclass[11pt]{article}

\usepackage[T1]{fontenc}
\usepackage[utf8]{inputenc}
\usepackage[francais]{babel}
\usepackage{fancyhdr}
\usepackage{datetime} % access to \currenttime
\usepackage{hyperref}
\usepackage[chorded]{songs}

\setlength{\parskip}{2ex}

\title{\vspace{-5em}Les accords du ukulele}
\author{(accordage : 4 cordes aigues de la guitare)}
\date{} 

\pagestyle{fancy}
\fancyhead[R]{\today~-~\currenttime}
\fancyfoot[C]{\thepage}

\begin{document}

\maketitle

\setcounter{tocdepth}{2}
\tableofcontents

\section{À propos de ce document}

Ce document est une présentation de quelques accords du ukulele accordé comme
les quatre cordes aigues de la guitare.

Il comporte deux tableaux :

\begin{itemize}
\item un où les accords sont organisés par fondamentale et couleur,
\item un autre où ils sont organisés par tonalité et degrés.
\end{itemize}

Attention, ce n’est ni un cours de ukulele, ni un cours d’harmonie.
Ça serait plutôt un support pour un cours de ukulele. On n’y explique pas en
détail la position des notes sur le manche, les doigtés, le rôle des notes dans
l’accord, les cadences… Pour cela, voir la section \ref{approf}.

On se restreint à cet accordage (non standard) : Ré, Sol, Si et Mi aigu
(de la quatrième corde à la première corde).

Voir la section \ref{contact} à la fin de ce document pour plus de détails ou
pour partager vos idées.

\section{Par fondamentale et couleur}

\begin{tabular}{ | c | c | c | c | c | c | c | c | c | }
    \hline
    Coul. \textbackslash Fonda.  & Do & Ré & Mi & Fa & Sol & La & Si \\
    \hline
    M  &
      {\gtab{}{2010:3513}} {\gtab{}{2013:3515}} &
      {\gtab{}{0232:1513}} &
      {\gtab{}{2100:1351}} &
      &
      {\gtab{}{0003:5131}} &
      {\gtab{}{2220:5135}} &
      \\
    \hline

    m  & 
      {\gtab{}{1010:3513}} {\gtab{}{1013:3515}} &
      {\gtab{}{0231:1513}} &
      {\gtab{}{2000:1351}} &
      &
      voir G5 &
      {\gtab{}{2210:5135}} &
      {\gtab{}{0402:3115}}
      \\
    \hline

    7  &
      {\gtab{}{2310:3713}} {\gtab{}{2313:3715}} &
      {\gtab{}{0212:1573}} &
      {\gtab{}{0100:7351}} &
      &
      {\gtab{}{0001:5137}} &
      {\gtab{}{2020:5735}} &
      {\gtab{}{1202:3715}}
      \\
    \hline

    M7 &
      {\gtab{}{2000:3573}} &
      {\gtab{}{0(222):1573}} &
      &
      {\gtab{}{3210:1357}} &
      {\gtab{}{0002:5137}} &
      {\gtab{}{2120:5735}} &
      \\
    \hline

    m7 &
      {\gtab{}{1313:3715}} &
      {\gtab{}{0211:1573}} &
      {\gtab{}{0000:7351}} &
      &
      &
      {\gtab{}{2010:5735}} &
      {\gtab{}{0202:3715}}
      \\
    \hline

    m7$\flat5$ &
      {\gtab{}{1312:3715}} &
      {\gtab{}{0111:1573}} &
      {\gtab{}{2333:1573}} &
      {\gtab{}{1101:7351}} &
      {\gtab{}{3323:7351}} &
      {\gtab{}{1213:5137}} &
      {\gtab{}{0201:3715}}
      \\
    \hline

    misc &
      &
      {\gtab{Ré 7$\flat$9}{1212:9573}} &
      {\gtab{Mi 7$\flat$9}{0101:7359}} &
      &
      {\gtab{Sol 5}{0033:5151}} &
      {\gtab{La 7$\flat$9}{2323:5937}} &
      {\gtab{Si 7$\flat$9}{1212:3795}}
      \\
    \hline
\end{tabular}

\section{Par tonalité et degré}

\begin{tabular}{ | c | c | c | c | c | c | c | c | c | }
    \hline
    Do  / La m &
      I M7 & II m7 & III m7 & IV M7 & V 7 & VI m7 & VII m7$\flat$5 \\
    \hline
    3 sons &
      {\gtab{Do}{2010:3513}} {\gtab{Do}{2013:3515}} &
      {\gtab{Ré m}{0231:1513}} &
      {\gtab{Mi m}{2000:1351}} &
      {} &
      {\gtab{Sol}{0003:5131}} &
      {\gtab{La m}{2210:5135}} &
      \\
    \hline
    4 sons &
      {\gtab{Do M7}{2000:3573}} &
      {\gtab{Ré m7}{0211:1573}} &
      {\gtab{Mi m7}{0000:7351}} &
      {\gtab{Fa M7}{3210:1357}} &
      {\gtab{Sol 7}{0001:5137}} &
      {\gtab{La m7}{2010:5735}} &
      {\gtab{Si m7$\flat$5}{0201:3715}}
      \\
    \hline
    autres &
      {\gtab{Do 7}{2310:3713}} {\gtab{Do 7}{2313:3715}} &
      {\gtab{Ré 7}{0212:1573}} &
      {\gtab{Mi 7}{0100:7351}} &
      {} & {} &
      {\gtab{La 7}{2020:5735}} &
      \\
    \hline
\end{tabular}

\begin{tabular}{ | c | c | c | c | c | c | c | c | c | }
    \hline
    Ré  / Si m &
      I M7 & II m7 & III m7 & IV M7 & V 7 & VI m7 & VII m7$\flat$5 \\
    \hline
    3 sons &
      {\gtab{Ré}{0232:1513}} &
      {\gtab{Mi m}{2000:1351}} &
      {} &
      {\gtab{Sol}{0003:5131}} &
      {\gtab{La}{2220:5135}} &
      {\gtab{Si m}{0402:3115}} &
      \\
    \hline
    4 sons &
      {\gtab{Ré M7}{0(222):1573}} &
      {\gtab{Mi m7}{0000:7351}} &
      {} &
      {\gtab{Sol M7}{0002:5137}} &
      {\gtab{La 7}{2020:5735}} &
      {\gtab{Si m7}{0202:3715}} &
      \\
    \hline
    autres &
      {\gtab{Ré 7}{0212:1573}} &
      {\gtab{Mi 7}{0100:7351}} &
      {} & {} &
      {\gtab{La m7}{2010:5735}} &
      {\gtab{Si 7}{1202:3715}} &
      \\
    \hline
\end{tabular}

\begin{tabular}{ | c | c | c | c | c | c | c | c | c | }
    \hline
    Mi  / C$\sharp$ m & 
      I M7 & II m7 & III m7 & IV M7 & V 7 & VI m7 & VII m7$\flat$5 \\
    \hline
    3 sons &
      {\gtab{Mi}{2100:1351}} &
      & {} &
      {\gtab{La}{2220:5135}} &
      & {} & \\
    \hline
    4 sons &
      {} & {} & {} &
      {\gtab{La M7}{2120:5735}} &
      {\gtab{Si 7}{1202:3715}} &
      &\\
    \hline
    autres &
      {\gtab{Mi 7}{0100:7351}} &
      {} & {} &
      {\gtab{La m}{2210:5135}} &
      {\gtab{Si m7}{0202:3715}} &
      &\\
    \hline
\end{tabular}

\begin{tabular}{ | c | c | c | c | c | c | c | c | c | }
    \hline
    Fa  / Ré m &
      I M7 & II m7 & III m7 & IV M7 & V 7 & VI m7 & VII m7$\flat$5 \\
    \hline
    3 sons &
      {\gtab{Fa M7}{3210:1357}} &
      {\gtab{G5}{0033:5151}} &
      {\gtab{La m}{2210:5135}} &
      &
      {\gtab{Do}{2010:3513}} {\gtab{Do}{2013:3515}} &
      {\gtab{Ré m}{0231:1513}} &
      \\
    \hline
    4 sons &
      {} & {} &
      {\gtab{La m7}{2010:5735}} &
      {} &
      {\gtab{Do 7}{2310:3713}} {\gtab{Do 7}{2313:3715}} &
      {\gtab{Ré m7}{0211:1573}} &
      {\gtab{Mi m7$\flat$5}{2333:1573}}
      \\
    \hline
    autres &
      {} &
      {\gtab{Sol 7}{0001:5137}} &
      {\gtab{La 7}{2020:5735}} &
      {} &
      {\gtab{Do m}{1010:3513}} {\gtab{Do m}{1013:3515}} &
      {\gtab{Ré 7}{0212:1573}} &
      \\
    \hline
\end{tabular}

\begin{tabular}{ | c | c | c | c | c | c | c | c | c | }
    \hline
    Sol / Mi m & 
      I M7 & II m7 & III m7 & IV M7 & V 7 & VI m7 & VII m7$\flat$5 \\
    \hline
    3 sons &
      {\gtab{Sol}{0003:5131}} &
      {\gtab{La m}{2210:5135}} &
      {\gtab{Si m}{0402:3115}} &
      {\gtab{Do}{2010:3513}} {\gtab{Do}{2013:3515}} &
      {\gtab{Ré}{0232:1513}} &
      {\gtab{Mi m}{2000:1351}} &
      \\
    \hline
    4 sons &
      {\gtab{Sol M7}{0002:5137}} &
      {\gtab{La m7}{2010:5735}} &
      {\gtab{Si m7}{0202:3715}} &
      {\gtab{Do M7}{2000:3573}} &
      {\gtab{Ré 7}{0212:1573}} &
      {\gtab{Mi m7}{0000:7351}} &
      \\
    \hline
    autres &
      {\gtab{Sol 7}{0001:5137}} &
      {\gtab{La 7}{2020:5735}} &
      {\gtab{Si 7}{1202:3715}} &
      {\gtab{Do m}{1010:3513}} {\gtab{Do m}{1013:3515}} &
      {\gtab{Ré m}{0231:1513}} &
      {\gtab{Mi 7}{0100:7351}} &
      \\
    \hline
\end{tabular}

\begin{tabular}{ | c | c | c | c | c | c | c | c | c | }
    \hline
    La  / Fa$\sharp$ m &
      I M7 & II m7 & III m7 & IV M7 & V 7 & VI m7 & VII m7$\flat$5 \\
    \hline
    3 sons &
      {\gtab{La}{2220:5135}} &
      {\gtab{Si m}{0402:3115}} &
      &
      {\gtab{Ré}{0232:1513}} &
      {\gtab{Mi}{2100:1351}} &
      {} &\\
    \hline
    4 sons &
      {\gtab{La M7}{2120:5735}} &
      {\gtab{Si m7}{0202:3715}} &
      {} &
      {\gtab{Ré M7}{0(222):1573}} &
      {\gtab{Mi 7}{0100:7351}} &
      {} &\\
    \hline
    autres &
      {\gtab{La 7}{2020:5735}} &
      {\gtab{Si 7}{1202:3715}} &
      {} &
      {\gtab{Ré m}{0231:1513}} &
      {\gtab{Mi m}{2000:1351}} &
      {} &\\
    \hline
\end{tabular}

\section{Approfondissement} \label{approf}

Je propose justement dans mes cours de comprendre les gammes, la construction
des accords, le rôle des notes, des idées d’utilisation de ces accords (disons
en gros~: la main droite et des morceaux précis).

Ne pouvant pas faire mieux faute de temps, je vous propose ici d’utiliser un
moteur de recherche pour y trouver ces informations. Par exemple~:

\begin{itemize}
\item la gamme majeure~;
\item l’harmonisation de la gamme majeure~;
\item les autres gammes~;
\item des vidéos sur les rythmiques ou battements à la guitare~;
\item ne pas hésiter à chercher «~apprendre \textit{[ici mon morceau préféré]}
       à la guitare~».
\end{itemize}

\section{Contact} \label{contact}

\setlength{\parindent}{0pt}

Site~: \url{https://grahack.github.io/accords_ukulele/} \\
ou plus simplement \url{https://huit.re/acc-uke}

Email~: \url{profgra.org@gmail.com}

\end{document}
